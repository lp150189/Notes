%%This is a very basic article template.
%%There is just one section and two subsections.
\documentclass{article}

\begin{document}


\section{480 Note, May 1st 201}

\subsection{Introduction}
\textbf{ 60 \% of a program life time comes from}
\begin{itemize}
  \item feature change
  \item bugs
\end{itemize}

\textbf{ how to improve your code. using the Quality Attributes}
\begin{itemize}
  \item Flexibility
  \item Reusability
\end{itemize}

- These two attributes hold the highest positions in all the attributes list

\textbf{How do I know if the design is reusable}\\
- We will use the metric to measure that
\begin{itemize}
  \item Coupling
  \item Cohesion
  \item Corresponding
\end{itemize}

\subsection{ Metrics }
- Measure for how much a class is intra-related
- Example:\\
- Hotel Reservation class:
\underline{ Fields}
\begin{itemize}
  \item Date stat
  \item Date rad
  \item String name
  \item int billingCard
  \item Room roomtype
\end{itemize}
\underline{method}
\begin{itemize}
  \item reservation(Date, int, int)
  \item checkin()
  \item checkout
  \item addRoomCharges
\end{itemize}

\textbf{Graph that show the relations between fields and methods}\\

\textbf{ What is the benefit of class doing one thing --Answer: you can reuse it
easlity, if you have a bug, you know where to find it}\\

\textbf{Cohesion}\\
\text{Definition:???}\\
- Low cohesion is bad\\
- High cohesion is good\\

\textbf{Coupling}\\
\textbf{Definition: How much are classes related, or depend on each other}\\
-\underline{ Example:}\\
if we have classes of A,B,C,D,E and they all related to each other. If we modify
one class, we will break everything\\
-\underline{Another example:} HTML is a standard, and Microsoft has an extension
of HTML, and only IE understands it, you write a program that IE can only
understand it. So your website is tied to Microsoft, if Microsoft changes
anything, you have to change it too.\\
- \textbf{With Coupling we want:}
\begin{itemize}
  \item high coupling is very bad
  \item low coupling is very good
\end{itemize}

- \textbf{Conclusion:} To make the program good(flexible ,a d Reusability): we
need to aim for low coupling and high cohesion.\\

\textbf{Example for low coupling or more flexible:} void
foo(ArrayList<Integer>):
this is not good because it is too narrow, so instead of using ArrayList, we can use List, or Collection which
is the interface. So that when you change the passed down object, you don't have
to change too much.\\

\textbf{Correspondence:}\\
\textbf{Definition:} How much the require match the requirements. How much it
match the change of requirement\\
\textbf{Example:} if you ask web designer for minor change, and it takes too
much time and effor, then the design is very bad. The programmer didn't think
much about the changes.\\
\textbf{ Good Correspondence}\\
\begin{itemize}
\item Minor requirement changes only needs low cost in design\\
\item Major changes changes need high cost to design\\
\end{itemize}





\subsection{Design Principles}
\begin{itemize}
  \item Encapsulation
  \item Information Hiding
  \item Abstraction
  \item Inheritance
  \item Polymorphism
  \item Substitution
\end{itemize}

\textbf{Encapsulation}\\
\textbf{Definition:} is the principle of bundling together the data and code
that works on that data.\\
\textbf{Example:} A class that has public fields, public methods. That's
encapsulation but no information hiding. In C, global variables are
encapsulation but not information hiding.

\textbf{Information Hiding}: is the principle of hiding the design decisions
which are most likely to change
\textbf{Example:} put every fields in your class to be private. Sort( int[]
arr)\\
\textbf{Abstraction}: the priciple of describing the essiential features of a
type without the inessential details. 
\textbf{Example:} List oppose to ArrayList, Parent object as oppose to the
subobject
\textbf{question:} In java, we use classes to do encapsulation and information
hiding
\textbf{Note}: you always tend to lean to information hiding for a better
solution

\textbf{Inheritance and Polymorphsm}: They are very close term. Inheritance
means sub-class inherit methods and fields from the parent class. Polimorphsm is
more like when you can have an array of subtract class(Animal ), in that array,
there are many many type of animals like dog, cat, turtle. 
\textbf{Substitution}: ???
\textbf{Pre and post condition is a very important part in Interface.}

\begin{center}
\textsc{\LARGE{ SO TIRED }}
\end{center}

\textbf{Example picture:}



\textbf{}
\end{document}
