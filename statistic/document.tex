%%This is a very basic article template.
%%There is just one section and two subsections.
\documentclass{article}

\begin{document}


\section{Statistic}

\subsection{Subtitle}
Tuesday, April 23, 2013.\\

Chapter 5: Probability Desities\\
	For continous random variable, the outcome are represented by
	intervals on the real line.\\
	Interval [-------------------] has infinite number of value\\
	The probability distribution for a continous random variable is 
	represented by the area under a curve $f(x)$ => probability density 
	function\\
	Discrete $ P(\bar{X} = x)$ continous when $P(a<=\bar{X}<=b)$ =
	``Intergration from a to b'' f(x)dx\\
	If you try to find $ P(\bar{X} = x)$  = ``intergration x to x'' f(x)dx=0
	, then => the probability of a specfic value =0\\
	However we are only looking at intervals
	$P(a<=\bar{X}<=b)$  = $P(a<\bar{X}<b)$  with at a and b = 0\\
	
	The probability of a function from value a to b will be the area below 
	the curve of the function from a to b\\
	\\
	Cumulative probability functions\\
	$F(x) = P(\bar{X} <= x)$ = ``integral from -infinity to x `` f(t)dt\\
	
	Example: Probability density function\\
	$ f(x) = e^{-x} when x>=0  or 0  x<0$\\
	
	$P(1<=\bar{X}<=5)$ = ``integral 5 to 1'' $e^{x}dx$ = .239\\
	
	Example: Probability density function
	
	$f(x) = 0  when x <0|| 1/2 when 0<x<1 || 2-x when 1<x<2|| 0 when x> 2$
	
	$P(.5<=\bar{X}<=1.5) = 5/8$
	
	Mean of the Probability density
	weighted average\\
	M = \[ \int_{big}^{big}xf(x)\,dx.\] 
	
	Variance of probability density \\
	\sigma^2 =  \[ \int_{big}^{big}(x-\mu)^2f(x)\,dx.\]


\subsection{5.2 Normal Distribution}
	1)area uner curve = 1\\
	2) highes pt at mean\\
	3) Symmetric about \mu\\
	4) mean = median = mode\\
	5)Each unique normal distribution can describe by \mu, \sigma \\
	
	$f(x, \mu, \sigma) = \frac{1}{2\pi}$

\end{document}
