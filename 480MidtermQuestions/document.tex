%%This is a very basic article template.
%%There is just one section and two subsections.
\documentclass{article}

\begin{document}


\section{Questions for Test}

\textbf{Question 1:}
What types of errors are you likely to miss even with 100\%  level
coverage? Provide two examples and explain

\textbf{Answer:}
\begin{itemize}
  \item Performance Defecs: Even if you have 100\% statment level coverage,
  performance is something that you will likely to miss because the program is
  required to be tested with a very large input in order to see what performance
  it has.
  \item Requirement bugs: bugs are come from missunderstanding the requirement
  from the program can't be detected even with 100\% coverage, these bugs can
  only be worked out with discussion among the team, project manager, and the
  customer in general.
\end{itemize}

\textbf{Question 2:} What kind of error are surfaced by regression testing.\\
\textbf{Answer:} Usually human errors in revision. Usually when a programmer
revive a program, chances are some of the old errors will re-occur. Regression
will prevent these error effectively.\\


\textbf{Question 3:} Error, failure, faults, hazard. Describe the four concepts
when communicating about a defects. Describe the four concepts and give example
illustrating the differences. \\
\textbf{Answer:}\\
\textbf{Error:} dynamic problem at runtime. \underline{ Eg:} Index out of bounds
in arrays.\\
\textbf{Failure:} is the problem of the system or the machine. \underline{ Eg:}
Therac 25 believes it has valid data when it doesn't and shoot overdose stream
of radiation.\\
\textbf{Fault:} static problem in code. \underline{Eg: } increment a field
instead of setting it to a particular value\\
\textbf{ Hazard: } The resulting risk that actually happens. \underline{Eg: }
patient dies because of therac 25 radiation overdose\\

\textbf{Question 4:} How does unit testing improve reliability? Use terminology
above when answering this question\\
\textbf{Answer:} Unit testing prevent faults which leads to error from
happening( testing at different level like method, single class, entire system).
And since faults is the lowest in the progression of defects, unit testing
strongly improve reliability. Progression of defects:\\
\begin{itemize}
  \item Fault
  \item Error
  \item Failure
  \item Hazard
  
  \end{itemize}
  
  \textbf{Question 5}: 4 stages of team dynamics, briefly describe the
  characteristic of each\\
  \textbf{Answer:} 
  \begin{itemize}
    \item \textbf{Forming:} People try to get to know each other, low conflict,
    low production
    \item \textbf{Storming:} People working our roles, and find out weakness and
    strength. Low production, high conflict
    \item  \textbf{Norming:} group learns how to deal with conflict, become more
    productive with defined roles. medium conflict, medium production.
    \item  \textbf{Performing:} Team member understands each other, team's
    culture is formed. Conflicts are dealt in a way that does not cause personal
    upset. High production, medium conflicts
  \end{itemize}

\textbf{Question 6:} Compare Inspection to Analysis\\
\textbf{1.} Give an example of a defect found more easily with static analysis
than inspection, and explain why it is better found with static analysis\\
\textbf{Answer:} Security errors such as buffer overflows are difficult to find
during inspection because it's hard to look or trace code by hand and easier to
detect via static analysis tools(design to analyize control and data flow)\\
\textbf{2.}Give an example of a defect found more eaisly with inspection that
static analysis, explain why it is better found with inspection\\
\textbf{Answer:} Issues of scalability are much easier to identify and discuss
as a team during inspection as opposed to static testing. For instance, more
efficient algorithms (complexity analysis) may be suggested during inspections
 whereas a static analysis is not designed to look for such things.\\
 





\end{document}
